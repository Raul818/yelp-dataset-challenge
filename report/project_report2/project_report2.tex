%%%%%%%%%%%%%%%%%%%%%%%%%%%%%%%%%%%%%%%%%%%%%%%%%%%%%%%%%%%%%%%%%%
%%%%%%%% ICML 2015 EXAMPLE LATEX SUBMISSION FILE %%%%%%%%%%%%%%%%%
%%%%%%%%%%%%%%%%%%%%%%%%%%%%%%%%%%%%%%%%%%%%%%%%%%%%%%%%%%%%%%%%%%

% Use the following line _only_ if you're still using LaTeX 2.09.
%\documentstyle[icml2015,epsf,natbib]{article}
% If you rely on Latex2e packages, like most moden people use this:
\documentclass{article}

% use Times
\usepackage{times}
% For figures
\usepackage{graphicx} % more modern
%\usepackage{epsfig} % less modern
\usepackage{subfigure}
\usepackage{amsmath}
\usepackage{amsthm}
\usepackage{amssymb}
\theoremstyle{plain}
\newtheorem{thm}{Theorem}[section]
\newtheorem{lem}[thm]{Lemma}
\newtheorem{prop}[thm]{Proposition}
\newtheorem*{cor}{Corollary}
\theoremstyle{definition}
\newtheorem{defn}{Definition}[section]
\newtheorem{conj}{Conjecture}[section]
\newtheorem{exmp}{Example}[section]
\theoremstyle{remark}
\newtheorem*{rem}{Remark}
\newtheorem*{note}{Note} 

% For citations
\usepackage{natbib}

% For algorithms
\usepackage{algorithm}
\usepackage{algorithmic}

% As of 2011, we use the hyperref package to produce hyperlinks in the
% resulting PDF.  If this breaks your system, please commend out the
% following usepackage line and replace \usepackage{icml2015} with
% \usepackage[nohyperref]{icml2015} above.
\usepackage{hyperref}

% Packages hyperref and algorithmic misbehave sometimes.  We can fix
% this with the following command.
\newcommand{\theHalgorithm}{\arabic{algorithm}}

% Employ the following version of the ``usepackage'' statement for
% submitting the draft version of the paper for review.  This will set
% the note in the first column to ``Under review.  Do not distribute.''
\usepackage[accepted]{icml2015} 

% Employ this version of the ``usepackage'' statement after the paper has
% been accepted, when creating the final version.  This will set the
% note in the first column to ``Proceedings of the...''
%\usepackage[accepted]{icml2015}


% The \icmltitle you define below is probably too long as a header.
% Therefore, a short form for the running title is supplied here:
\icmltitlerunning{Submission and Formatting Instructions for ICML 2015}

\begin{document} 

\twocolumn[
\icmltitle{Project report 2 for CMPS 242 \\ 
           Machine Learning Fall 2016}

% It is OKAY to include author information, even for blind
% submissions: the style file will automatically remove it for you
% unless you've provided the [accepted] option to the icml2015
% package.
\icmlauthor{Greeshma Swaminathan}{gswamina@ucsc.edu}
\icmlauthor{Neha Ojha}{nojha@ucsc.edu}
\icmlauthor{Jianshen Liu}{jliu120@ucsc.edu}
\icmlauthor{Alex Bardales}{abardale@ucsc.edu}
\icmladdress{University of California, Santa Cruz,
           1156 High Street, Santa Cruz, CA 95064} 

% You may provide any keywords that you 
% find helpful for describing your paper; these are used to populate 
% the "keywords" metadata in the PDF but will not be shown in the document
\icmlkeywords{boring formatting information, machine learning, ICML}

\vskip 0.3in
]

\section{Proposed project details}
For the final project we will implement one of the algorithms discussed in the article ``Recommendation in Heterogeneous Information Networks with Implicit User Feedback,'' by Yu, et. al. The article is from 2014 and it uses the Yelp dataset as one of its test cases. Several machine learning algorithms (Popularity, Co-Click, NMF, and Hybrid-SVM) against the authors' reported algorithm, HeteRec. Upon further investigation into these machine learning algorithms, we will choose one to implement.

Before we present the problem statement, we need several definitions, the first of which is called an ``Information network'', [from~\ref{}]:


\begin{defn}[Information network]
An information network si defined as a directed graph $G = (V,E)$ with an entity type mapping function $\Phi:V \rightarrow \mathcal{A}$ and a link type mapping function $\Psi : E\rightarrow \mathcal{R}$ where each entity $v\in V$ belongs to one particular entity type $\Phi(v)\in \mathcal{A}$, and each link $e \in E$ belongs to a particular relation type $\Psi(e) \in \mathcal{R}$. When the types of entities $|\mathcal{A}| > 1$ and the types of relations $|\mathcal{R}| > 1$, then the network is called \textit{heterogeneous information network}.
\end{defn}

Next we present a user feedback representation, from ~[\ref{}].

\begin{defn}[Binary User Feedback]
With $m$ users $\mathcal{U} = \{ u_1, \dots,u_m\}$ and $n$ items $\mathcal{I} = \{e_1, \dots,e_n\}$, we define the binary user feedback matrix $R \in \mathbb{R}^{m\times n}$ as
$
\begin{displaystyle}
  R_{ij} = 
  \begin{cases}
    1,  & \text{if } (u_i, e_j) \text{ interaction is observed;}\\
    0,  & \text{otherwise.}\\
  \end{cases}
\end{displaystyle}$
\end{defn}


With these definitions, we are able to present the problem statement from the article.

\begin{defn}[Problem Statement]
Given a binary rating matrix $R$, and a related heterogeneous information network $G$, for a user $u_i \in \mathcal{U}$, we aim to recommend a ranked list of items $I_{u_i} \subset \mathcal{I}$ that are of interests to this user. 
\end{defn}

The article presents a user preference diffusion based feature generation method.

% Acknowledgements should only appear in the accepted version. 
%\section*{Acknowledgments} 
 


% In the unusual situation where you want a paper to appear in the
% references without citing it in the main text, use \nocite
\nocite{}

\bibliography{}
\bibliographystyle{icml2015}

\end{document} 


% This document was modified from the file originally made available by
% Pat Langley and Andrea Danyluk for ICML-2K. This version was
% created by Lise Getoor and Tobias Scheffer, it was slightly modified  
% from the 2010 version by Thorsten Joachims & Johannes Fuernkranz, 
% slightly modified from the 2009 version by Kiri Wagstaff and 
% Sam Roweis's 2008 version, which is slightly modified from 
% Prasad Tadepalli's 2007 version which is a lightly 
% changed version of the previous year's version by Andrew Moore, 
% which was in turn edited from those of Kristian Kersting and 
% Codrina Lauth. Alex Smola contributed to the algorithmic style files.  
